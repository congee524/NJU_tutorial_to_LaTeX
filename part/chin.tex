\section{中文支持}
\subsection{了解\CTeX}
\begin{frame}{How to get Chinese document?}{\CTeX}
	\CTeX 是最新的也是最便利的\LaTeX 中文支持的开源项目,可以参考其\href{material/LM/CTEXmanual.pdf}{\textcolor{blue}{使用手册}}。
	\begin{itemize}
		\item 全面汉化
		\item 更智能的排版
		\item 强大的自定义设置
	\end{itemize}
	一般使用默认设置即可,有特殊需求可翻阅手册作适当调整。
\end{frame}
\begin{frame}
	\begin{table}\caption{\CTeX 宏集的组成}
		\footnotesize
		\begin{tabular}{llp{6cm}}
			\toprule
			类别 & 文件 & 说明 \\
			\midrule
			文档类 & ctexart.cls & 标准文档类article 的汉化版本,一般适用于短篇幅的文章\\
			& ctexrep.cls & 标准文档类report 的汉化版本,一般适用于中篇幅的报告\\
			& ctexbook.cls & 标准文档类book 的汉化版本,一般适用于长篇幅的书籍\\
			& ctexbeamer.cls & 文档类beamer 的汉化版本,适用于幻灯片演示\\
			\midrule
			宏包 & ctex.sty & 提供全部功能,但默认不开启章节标题设置功能,需要使用heading 选项来开启\\
			& ctexsize.sty & 定义和调整中文字号,在ctex 宏包或CTEX 中文文档类之外单独调用\\
			& ctexheading.sty & 提供章节标题设置功能,在ctex 宏包或CTEX 中文文档类之外单独调用\\
			\bottomrule
		\end{tabular}
	\end{table}
\end{frame}
\begin{frame}[fragile]
	\begin{itemize}
		\item 文档类
\begin{lstlisting}
\documentclass{ctexart}
\documentclass{ctexrep}
\documentclass{ctexbook}
\documentclass{ctexbeamer}
\end{lstlisting}		
		\item 宏包
\begin{lstlisting}
\usepackage{ctex}
\usepackage{ctexsize}      
\usepackage{ctexheading}   
\end{lstlisting}
	\end{itemize}
	\vspace{2ex}
	\textcolor{red}{注意}:\texttt{ctexcap} 是默认打开 \texttt{heading} 选项的 \texttt{ctex} 宏包,手册中明确说明其为过时宏包!
\end{frame}
\begin{frame}
	两个最常用选项\texttt{heading}与\texttt{scheme}。

	打开\texttt{ctex.tex},体会不同选项的区别。
	\begin{itemize}
		\item \texttt{ctexart}类文档与\texttt{ctex}宏包的默认选项都是\texttt{heading=true,scheme=chinese}
		\item \texttt{heading=false,scheme=plain}相当于只是支持中文,所有的版式都还是英文的
		\item 如果觉得标题居中太蠢了想要靠左,可以用\texttt{heading=false}(\texttt{scheme=chinese}是默认选项不用改)
	\end{itemize}
\end{frame}