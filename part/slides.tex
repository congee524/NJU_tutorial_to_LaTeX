\section{来做幻灯片}
\begin{frame}
	诚如本讲义所展示的,使用\LaTeX 的\texttt{beamer}文档类型可以制作幻灯片。其在语法上与普通的文档别无二致,唯一区别在于组成的基本单位变为了\textbf{帧}(frame)。
\end{frame}

\subsection{帧}
\begin{frame}[fragile]{一帧即为一页}
	\begin{itemize}
		\item 每一页的内容都是以 \verb|\begin{frame}|开始,以 \verb|\end{frame}|结尾
		\item 如果内容简短,也可以采用\verb|\frame{<text>}|的形式
		\begin{itemize}
			\item \verb|\frame{\titlepage} %标题页| 
			\item \verb|\frame{\tableofcontents} %目录页|
		\end{itemize}
	\end{itemize}	
\end{frame}
\begin{frame}[fragile]{标题页}
	\begin{itemize}
		\item 与\texttt{article}类文档等用法相同,只是相比而言更为详细
		\item 以本讲义的标题页为例
\begin{lstlisting}
\title[\LaTeX 教程]{\LaTeX ~Tutorials}
\subtitle{Simple and professional}
\author[高嵩,黄秉焜]{Song Gao, Bingkun Huang}
\institute[南京大学]{Nanjing University}
\date[2018.10.28]{\today}
\logo{\includegraphics[width=15pt]{NJU.jpg}}
\end{lstlisting}
		\item 方括号内为短标题,显示位置视主题模板而定,大括号内为长标题,显示于标题页
	\end{itemize}
\end{frame}
\begin{frame}[fragile]{文档层次与目录}
	\begin{itemize}
		\item 一般最高为 \verb|\section|,其次为 \verb|\subsection|,而 \verb|\subsubsection|及以下很少使用
		\item 使用\verb|\tableofcontents|生成目录
		\item 有可选项:\verb|hideallsubsection|和\verb|currentsection|等
		\item 与\verb|\AtBeginSection[]{}|配合使用
\begin{lstlisting}
\AtBeginSection[]
{
  %\begin{frame}
    \tableofcontents[currentsection,hideallsubsections]
  %\end{frame}
}
\end{lstlisting}
	\end{itemize}
\end{frame}
\begin{frame}[fragile]{帧标题}
	\begin{itemize}
		\item 每一帧都有一个\texttt{frametitle}和一个\texttt{framesubtitle}
		\item 标准使用格式是这样的
\begin{lstlisting}
%begin{frame}
\frametitle{标题}
\framesubtitle{副标题}
%内容
%\end{frame}
\end{lstlisting}
		\item 这太麻烦了,可以简写成
\begin{lstlisting}
%\begin{frame}{标题}{副标题}
%内容
%\end{frame}
\end{lstlisting}
		\item 标题和副标题都是可选项
	\end{itemize}
\end{frame}
\begin{frame}[fragile]{最常用的环境——\texttt{itemize}}
	\begin{itemize}
		\item 这是第一层
		\begin{itemize}
			\item 这是第二层
			\begin{itemize}
				\item 这是第三层
				\item 没有第四层
			\end{itemize}
			\item 超出三层会报错
		\end{itemize}
	\end{itemize}
	\small
\begin{lstlisting}
\begin{itemize}
  \item 这是第一层
  \begin{itemize}
    \item 这是第二层
    \begin{itemize}
      \item 这是第三层
      \item 没有第四层
    \end{itemize}
    \item 超出三层会报错
  \end{itemize}
\end{itemize}
\end{lstlisting}
\end{frame}
\subsection{选择主题}
\begin{frame}[fragile]{主题}
	\begin{itemize}
		\item \verb|\usetheme{ }| 选择主题
		\item \verb|\usecolortheme{ }| 选择颜色主题
		\item \href{https://hartwork.org/beamer-theme-matrix/}{\textcolor{blue}{这个网页}}列出了所有beamer自带的主题与颜色主题
		\item \verb|\usefonttheme{ }| 还可以选择字体主题
		\item 按照国际惯例在slides中应使用无衬线字体,但是这种情况下的公式非常丑陋
		\item \verb|\usefonttheme{professionalfonts}|提供了解决方案,对于数学环境它将改用有衬线字体,正如前面的例子所示
	\end{itemize}
\end{frame}
\subsection{动态展示}
\begin{frame}[fragile]{项目依条出现}
	\begin{itemize}
		\item 这里给出一条最简单的动画显示的命令 \verb|\pause| \pause
		\item 得到的效果如此页所示 \pause
	\end{itemize}
\begin{lstlisting}
\begin{itemize}
  \item 这里给出一条最简单的动画显示的命令 \verb|\pause| \pause
  \item 得到的效果如此页所示 \pause
\end{itemize}
\end{lstlisting}
\end{frame}